% !TEX TS-program = pdflatex
% !TEX encoding = UTF-8 Unicode
\documentclass[a4paper]{article}
\usepackage[swedish]{babel}
\usepackage[T1]{fontenc}
\usepackage[utf8]{inputenc}
\usepackage{mathtools}
\usepackage[pdftex]{graphicx}
\usepackage{float}
\usepackage{fancyhdr}
\usepackage{geometry}
\usepackage{booktabs} % for much better looking tables
\usepackage{array} % for better arrays (eg matrices) in maths
\usepackage{paralist} % very flexible & customisable lists (eg. enumerate/itemize, etc.)
\usepackage{verbatim} % adds environment for commenting out blocks of text & for better verbatim
\usepackage{subfig} % make it possible to include more than one captioned figure/table in a single float

%%% HEADERS & FOOTERS

\author{Jonathan Karlsson - jonka293 - 890201-1991 \and Niclas Olofsson - nicol271 - 900904-5338}
\pagestyle{fancy} % options: empty , plain , fancy
\renewcommand{\headrulewidth}{1pt} % customise the layout...
\fancyhead[LO,LE]{Laboration 1 - TDDC78}
\lfoot{}\cfoot{\thepage}\rfoot{}
\setlength{\parindent}{0pt}

%%%% SECTION TITLE APPEARANCE

%\usepackage{sectsty}
%\allsectionsfont{\sffamily\mdseries\upshape} % (See the fntguide.pdf for font help)
%% (This matches ConTeXt defaults)
%
%%%% ToC (table of contents) APPEARANCE
%\usepackage[nottoc,notlof,notlot]{tocbibind} % Put the bibliography in the ToC
%\usepackage[titles,subfigure]{tocloft} % Alter the style of the Table of Contents
%\renewcommand{\cftsecfont}{\rmfamily\mdseries\upshape}
%\renewcommand{\cftsecpagefont}{\rmfamily\mdseries\upshape} % No bold!

%%% END Article customizations

%%% The "real" document content comes below...

\title{Laboration 1 - TDDC78}

%\date{} % Activate to display a given date or no date (if empty),
% otherwise the current date is printed

\begin{document}

\maketitle

\section{Program description}
\subsection{Threshold filter}

This program calculates the average intensity of the image and makes all
pixels with a higher-than-average value white, and all other pixels
black.\\

The whole image is read on the root node. The root node also calculates
the  average intensity of the image. The calcuated intensity is sent to
all other nodes via MPI broadcast.\\

The image data array is split in as many parts as there are nodes, and
each node gets its own part via MPI scatter. Since each node needs to
know  the data length to recieive, this is done in two steps where the
data length is  sent via MPI broadcast, and then sent with MPI scatter.
Each node then runs the threshold filter on its own part of the image.
MPI gather then reassembles the resulting image, which is written to
disk by the root node.\\

\subsection{Blur filter}

\section{Execution times}


\end{document}
